 Crystal orientation and distances: (FIND NICE IMAGES FOR THAT!!)
\begin{table}
\centering
\caption{Interatomic distances for Cu and Ag (both fcc) with respect to different surfaces. $a$ denotes the lattice constant and $\beta$ describes the angle within the (111) unit cell \SI{60}{\degree}}
  \begin{tabular}{ccccc}
& Lattice constant a [\SI{}{\angstrom}] & Nearest neighbours [\SI{}{\angstrom}] & diagonal [\SI{}{\angstrom}] \\ \hline \hline
\multicolumn{2}{c}{fcc(100)} & $\frac{\sqrt{2}a}{2}$ & a \\
  Cu	 	& 3.61	& 2.55 | 2.55 & 3.61 \\
  Ag		& 4.09	& 2.89 | 2.89 & 4.09 \\ \hline 
\multicolumn{2}{c}{fcc(110)} & $\frac{\sqrt{2}a}{2}$ | a & $\sqrt{\frac{3}{2}}a$\\
  Cu	 	& 3.61	& 2.55 | 3.61	& 4.42 \\
  Ag		& 4.09	& 2.89 | 4.09	& 5.00 \\ \hline 
\multicolumn{2}{c}{fcc(111)} & $\frac{\sqrt{2}a}{2} \ <110>$ & $\sqrt{2}a\sin(\frac{\beta}{2})$ | $\sqrt{2}a\cos(\frac{\beta}{2})$\\
  Cu 		& 3.61	& 2.55 | 2.55	& 2.55 | 4.42 \\
  Ag		& 4.09	& 2.89 | 2.89	& 2.89 | 5.01 \\ \hline
 \end{tabular}

\end{table}
 
 ``This surface consists of (111) terraces (three close-packed rows wide) and intrinsic (100) steps, which run parallel to the [011] direction. The close-packed atom rows located at the step edges are characterized by
a nearest-neighbour distance of \SI{2.55}{\angstrom}  for Cu and of \SI{2.89}{\angstrom} for Ag, whereas the intrinsic step
spacing is \SI{6.25}{\angstrom} for Cu and \SI{7.08}{\angstrom} for Ag. The surface symmetry is described by a primitive
rectangular unit cell (cf. Figure 3.1a). The (111) terraces and the microfacets which represent the intrinsic (100) steps are tilted by \SI{19.5}{\degree}, respectively, to macroscopic (211) surface, which can be seen in the side view of the hard-sphere model in the upper panel of Figure 3.1a. The interlayer spacing for this surface is \SI{0.74}{\angstrom} for Cu and \SI{0.83}{\angstrom} for Ag.''

Dense packed rows are for fcc(111) the followgin directions: $<\bar 1 01>$, $<01\bar 1>$, $<1\bar 1 0>$. The diagonals are found in the $<\bar 1 \bar 1 2>$ and $<1\bar 2 1>$ directions.

``The (311) surface consists also of (111) terraces (two close-packed rows wide) and intrinsic (100) steps.''

\begin{table}\label{tab:crystal-prop}
\caption{Crystal porperties from \cite[29ff]{riemann_ionic_2002} and \cite{liu_oxygen_2014}}
\centering
\begin{tabular}{cccc}
			&				& Copper 	 & Silver \\
\multicolumn{2}{c}{Lattice constant}			& \SI{3.61}{\angstrom} & \SI{4.08}{\angstrom} \\
\multicolumn{2}{c}{Nearest neighbour}			& \SI{2.55}{\angstrom} & \SI{2.89}{\angstrom} \\ \hline \\
\multirow{3}{*}{intrinsic step separation}	& (311) & \SI{4.23}{\angstrom} & \SI{4.78}{\angstrom} \\
						& (211) & \SI{6.25}{\angstrom} & \SI{7.07}{\angstrom} \\
						& (221) & \SI{7.66}{\angstrom} & \SI{8.65}{\angstrom} \\
						& (110) & \SI{1.38}{\angstrom} & \\
						& (111) & \SI{1}{\angstrom} & \\
\end{tabular}
\end{table}
Cu(100) has a step height of \SI{1.8}{\angstrom}. See http://dx.doi.org/10.1103/PhysRevB.94.064106
See \cite{riemann_ionic_2002} for another exmaples of vicinal metal surfaces (531),(532),(221),(311),(211).

The surface free energy increases from the (111) surface with increasing angle of the (hkl) surface of interest $$\cos(\phi)=\frac{h+k+l}{\sqrt{3(h^2+k^2+l^2)}}$$ \cite{jian-min_calculation_2004}. Thus, the (111) surface is the one with lowest energy, followed by (110) and (100).

\subsection{Dislocation lines and crystal orientation}
Due to the fact, that dislocastion lines move within the crystal in a defined manner, one can determine the crystals orientation when the orientation of a dislocation is known.
As for FCC crystals the orientation of dislocation lines occures in the {111} plane in $<110>$ direction. Its Burgers vector is $\frac{a}{2}[110]$\cite{_dislocation-theory}.