% TBP on Cu(111)
When adsorbed at room temperature, TBP distributes equally on the surface, forms unordered islands and decorates step edges. Molecules orient their main axis (connecting line from one di-tert-buytl-phenyl ring across the center to the nitrophenyl ring) along the dense packed substrate rows most often, less are \SI{15}{\degree} of. Several binding motivs (as shown in figure \ref{fig:binding-motivs-TBP-Cu111}) are observed, namely
\begin{itemize}
 \item A dimer, where molecules lie ``head-to-head'', functional groups ($NO_2$) pointing at each other
 \item A ``triangle'', where molecules are rotated \SI{120}{\degree} and functional groups point towards a shared center. Although this motiv does not occur very often (or at least under very flexible angles), it is given as an example where the functional groups point to each other. Similar motivs (like 3 molecules in \SI{90}{\degree} are observed together with other orientations. 
 \item Chains with different length appear, where the nitro-group of molecule 1 points to the di-tert-butyl group of molecule 2 (``head-to-tail''). At the connection points, molecules appear brighter, promoting a pyhsical overlap of the two molecules.
\end{itemize}

Center-center distances vary slightly, but is typically \SI{1.78}{\nano \meter} (for the head-to-tail) and \SI{1.5}{\nano \meter} for the head-to-head connection. 

\begin{figure}[ht]
 \centering
 \subfigure[Single leg nitro porphines adsorbed on Cu(111) surface at room temperature]{
  \includegraphics[width=0.4\textwidth]{./images/F151128-083339.jpg}
  }
 \subfigure[Model representation of the most observed binding motivs formed by TBP on Cu(111). See text for more details.]{
  \includegraphics[width=0.4\textwidth]{./images/TBP-motivs-on-Cu111}
  }
\caption{Adsorbed molecules and their model representation on the Cu(111) surface. Each of the binding motivs can be found as well in the STM data (a), as well as in the model respresentation (b).}
\label{fig:binding-motivs-TBP-Cu111}
\end{figure}

\paragraph{``head-to-head''}
To model the occuring binding motivs, deformations of the molecules have to be taken into account. Because nitro groups face each other in the ``head-to-head'' connection, their distance would be to small to faciliate a similar binding mechanism like for the TPCN on copper (where copper surface ad atoms promote binding between nitrogens), so no free space between the facing nitro-groups is observed. Because the distance is so small, the phenyl ring (with attached nitro group) rotates by \SI{45}{\degree}, to make the phenyl ring stand upright. When the second molecule does the same, both match each other with neglegible lateral shift, reproduing the STM images best. Similar binding motivs are reported in \cite{kato_dispersive_2008} for non-covalent crosslinking of dicarboxylic acids in hydrogels. Although the situation on a metal-surface may change considerably (only 2D - no 3D, metal present - will change chemistry), the observed binding motiv matches very well.

\paragraph{``head-to-tail''}
The chain motiv ``head-to-tail'' is reconstructed using the unique contrast of the TBP molecule. When the center-center distance is measured, molecules are modeled that distance away from each other. These models show a physical overlap between molecules, which in not possible because of steric hinderance. To solve the problem, the nitro-group (head) of one molecule if rotated by \SI{35}{\degree} out of the plane (like pulling the nitro-group upwards, not rotating the group left/right). 

%---------------- models bauen und bsp bilder einf\"ugen.  ---------------- 

Another interesting fact is that butyl groups of TBP seem to orient themself (as far as steric hinderance allows for) along the dense packed rows of the copper substrate. Again, one has to be careful when reconstructing geometrical information from STM images. Like the distortion of legs in the TPCN molecule, this rotation can be explained by a rotation of single butyl groups. Although the phenyl ring remains at the same position/rotation, tert-butyl groups are allowed to rotate such that they appear in different heights. Because STM (constant current) follows equipotential lines, the whole phenyl-di-tert-butyl-complex looks rotated in plane, although it may not. This is confirmed in literature\cite{heim_surface-assisted_2010,heim_self-assembly_2010}.

If this is the driving force for orienting the whole molecule on the surface remains speculative. On Ag(100), neither an orientation of the molecules main axis with respect to the substrate, nor a orientation of butyl-groups along the dense packed substrate rows can be seen - which again favours Cu-subtrate interactions as dominant role.