The bought and chemically polished foils are mounted on a sample holder and loaded into the load lock. It is evacuated for \SIrange{2}{3}{\hour} and the valve is opened to the chamber. During transfer, no noteable increase in the base pressure is noted. The sample is put on the parking slot.

The sample was initially degased with slowly increasing temperatures to remove adsorbats like $CO, CO_2$ and $H_2O$.

After some time of degassing, the sample was prepared with repeated sputter and anneal cycles. The annealing temperatures were increased up to \SI{800}{\degreeCelsius}. 
After that procedure, the sample was cooled down and observed in STM.

\begin{figure}[h!]
 \centering
 \includegraphics[width=0.5\textwidth]{./images/F150331-125720.jpg}
 \caption{Cu-foil after repeated sputtering and annealing cycles. The roughness is about \SI{72}{\pico\meter}}
 \label{fig:cu-foil-clean}
\end{figure}

A first look onto the sample shows a quite heterogen surface. While quite flat areas with a typical roughness of $\approx \SI{70}{\pico\meter}$ exist, there are also areas with very large corrugations $\geq \SI{100}{nm}$.