In the discussion of surface states, one generally distinguishes between Shockley states[5] and Tamm states,[6] named after the American physicist William Shockley and the Russian physicist Igor Tamm. However there is no real physical distinction between the two terms, only the mathematical approach in describing surface states is different. 

\paragraph{Shockley states - Electron gas approximation} from wikipedia article \\
Historically, surface states that arise as solutions to the Schr�dinger equation
in the framework of the nearly free electron approximation for clean and ideal surfaces, are called Shockley states. Shockley states are thus states that arise due to the change in the electron potential associated solely with the crystal termination. This approach is suited to describe normal metals and some narrow gap semiconductors. Figures 1 and 2 are examples of Shockley states, derived using the nearly free electron approximation.''

\paragraph{Tamm states - Tight binding approximation (LCAO)} from wikipedia article \\
Surface states that are calculated in the framework of a tight-binding model are often called Tamm states. In the tight binding approach, the electronic wave functions are usually expressed as linear combinations of atomic orbitals (LCAO). In contrast to the nearly free electron model used to describe the Shockley states, the Tamm states are suitable to describe also transition metals and wide gap semiconductors.

\paragraph{Extrinsic surface states} from wikipedia article \\
Surface states originating from clean and well ordered surfaces are usually called intrinsic. These states include states originating from reconstructed surfaces, where the two-dimensional translational symmetry gives rise to the band structure in the k space of the surface.

Extrinsic surface states are usually defined as states not originating from a clean and well ordered surface. Surfaces that match the category extrinsic are :
\begin{itemize}
 \item Surfaces with defects, where the translational symmetry of the surface is broken.
 \item Surfaces with adsorbates
 \item Interfaces between solid and liquid phases.
 \item Interfaces between two material such as a semiconductor-oxide or semiconductor-metal interfaces
\end{itemize}
Generally, extrinsic surface states cannot easily be characterized in terms of their chemical, physical or structural properties.

The \index{Kondo effect} \textbf{Kondo effect} also plays a role when looking at scattering of electrons on - in this special case - magnetic imputiries. For further reading one is advised to read \cite{kouwenhoven_revival_2001} and citations within.