\label{section:STS}
First changes of the tunneling current with the bias voltage were observed by Tromp et al. in 1986 \cite{tromp_atomic_1986}. They discovered a change in contrast when scanning a SI(111) surface with either positive or negative bias. The change in contrast is most apparent in semiconductors and semimetals\cite{bonnell_scanning_1993}, but adsorbats and charged areas of the sample change the DOS locally and therefore the contrast in STM.

The vanishing lateral movement of molecules in LT-STM makes them also accessible to tunneling spectroscopy. It is possible to deduce the electronic configuration on with atomic spatial resolution.
\paragraph{Modulating STM}
Spectroscopic information (information on the DOS) can be obtained by either changing the bias voltage [I(V,z)-spectroscopy] or the tip-sample distance [V(z)-spectroscopy].
While simple results may be already obtained when comparing two images recorded at different voltages, more detailed information can be achieved. A constant current image at a certain bias voltage is recorded. If the bias is then modulated with a sinus like waveform, one can extract the differential conductance ($\frac{dI}{dV}$). 
One provides a high frequency (much larger than feedback loop frequency of 1-2 kHZ), low amplitude modulation of the DC bias voltage. The AC part of the tunneling signal is than recorded with a lock in amplifier. The in-phase component is directly the $dI/dV|_{V=V_{bias}}$, recorded simultaniously with the topography. If the modulation frequency is too low, the feedback tries to compensate the modulation by changing the distance to the sample.
If the modulation frequency is too high, the capacitance between tip and sample leads to an $90\deg$ phase shifted current which increases with modulation frequency. One useally chooses the modulation freqeuncy slightly above the cutoff frequency for the feedback loop.

\paragraph{Bias below work function}
If tunneling conditions are such that $eV\leq\Phi$, observed features in $dI/dV$ are associated with the surface DOS. Critical points in the surface projected DOS give rise to features in $dI/dV$. Interpretation of these features with the WKB theory (i.e. differentiating equation \eqref{WKB}) gives
$$dI/dV=\rho_s(r,eV)\rho_t(r,0)T(eV,eV,r)+\int_0^{eV}\rho_s(eV)\rho_t(r,E-eV)\frac{dT(E,eV,r)}{dV}dE$$
The first term contains the DOS of the sample and tip and the transmission function. While it is usually unknown, a closer look to \eqref{Transmission-function} indicates a smooth, monotonically increasing function in V. This mannered dependence on V gives a smooth background described by the second term $\int_0^{eV}\rho_s(eV)\rho_t(r,E-eV)\frac{dT(E,eV,r)}{dV}dE$.
Because T is smooth and monotonic the first term $\rho_s(r,eV)\rho_t(r,0)T(eV,eV,r)$ introduces the dependence on the DOS in the sample for energies $eV$ - our desired spectrum.

If $dI/dV$ is recorded simultaniously with the topography, another contribution arises. One usually observes an decrease in atomic corrugation when the distance between tip and sample is increased. The surface looks flat. To have the same tunneling current on atom positions and in between, the decay length in the valleys $\kappa_v$ must be larger than on the atom positions $\kappa_a$. The Z-dependend corrugation given by Tersoff is $$\Delta(Z)\approx \frac{2}{\kappa}e^{-\frac{\pi^2Z}{a2\kappa}}$$ where a is the lattice constant and $\kappa$ the inverse decay length. To make both a flat looking surface one gets the expression
$$\kappa_v=\kappa_a-\frac{2\pi^2}{\kappa a^2}e^{-\frac{\pi^2\bar Z}{a2\kappa}}$$ 
As the transmission factor changes with the decay length, the tunneling current and with it the $dI/dV$ changes. This is the origin of topographic features in $dI/dV$ maps when recorded at constant current.

The origin of the strongly voltage dependend background can be found in WKB theory as well.
When writing the tunneling current as 
$$ I=\int_0^{eV}\rho_s(r,E)\rho_t(r,eV+E)exp\left(-\frac{2Z\sqrt{2m}}{\hbar}\sqrt{\frac{\Phi_s+\Phi_t}{2}+\frac{eV}{2}-E}\right)dE $$
the tunneling current reduces to 
\begin{equation}
\bar I=\rho_s\rho_t \bar V exp\left(-\frac{2\sqrt{2m}}{\hbar}\sqrt{\Phi}Z\right)
\label{tc}
\end{equation}
Assuming that DOS of tip and sample $\rho_t/\rho_s$ are constant, as well as discarding the change of the tunneling barrier with the bias voltage(an assumption only valid for very small voltages with $eV<<\Phi$) the derivative of \eqref{tc} is given by
$$\frac{dI}{dV}=e\rho_s\rho_texp\left(-\frac{2\sqrt{2m}}{\hbar}\sqrt{\Phi-\frac{eV}{2}}\right)Z$$
Substituting Z with the one obtained by \eqref{tc} leads to $dI/dV= \bar I / \bar V$ - which diverges as  $1/V$ when going to very low bias voltages and gives another contribution to the background. This makes it hard to observe features in close proximity to the fermi level ($V_{bias}=0V$). This background can be reduced when oparating at constant resistance and not at constant current. When doing this, features usually obscured by the $1/V$ diverging background can be observed.

For normalization and interpretation of data, see \cite{bonnell_scanning_1993}.
Normalization is done as argued by\cite{feenstra_tunneling_1987} because of the diverging part at \underline{\ \ \ \ \ \ 0V \ \ \ \ \ \ }.

\paragraph{Bias above work function}
If the bias voltage is higher than the work function of the sample $dI/dV$ reflects mainly states that arise from interaction of electrons at the surface with the polarization they induce in the bulk. Electrons are trapped by this interaction in a region near the surface leaving their lateral movement undistorted. These waves either do interfere con- or deconstructively at the surface. Which type of interference occurs is determined by the applied bias voltage that alternates the bounding condition. The transmission alternates when going from contructive to destructive interference and therefore the tunneling current changes when changing V. 
As an interesting fact, Becker et al.\cite{becker_electron_1985} found that that numerical integration of Schr\"odingers equation could be used together with $dI/dV$ spectra to calculate the absolute distance between tip and sample - an value hard to come by with other methods.

\paragraph{Barrier height measurements from I(Z) spectroscopy (work function difference)}\index{STS!Barrier Height}
Further information can be drawn from the tunneling system when the barrier height may be determined.
Taking the limit of the transmission function \eqref{Transmission-function} for low bias voltage ($eV\approx0$, $E=E_F$) results in 
$$T=exp\left(-\frac{2Z\sqrt{2m}}{\hbar}\sqrt{\frac{\Phi_s+\Phi_t}{2}}\right)$$
Using this in the WKB approximation \eqref{WKB}, one gets $$\frac{dI/dZ}{I}=\frac{2\sqrt{2m}}{\hbar}\sqrt{\Phi_s+\Phi_t}$$
As the work function of the tip usually stays constant, lateral variations in the barrier height can be boiled down to local changes in the work function. This is done by \cite{jia_variation_1998}.

Determining the barrier height in this way often results in to low values for the work function. Discussion of this is found in \cite[96]{bonnell_scanning_1993}.


\paragraph{Resolution}\index{STS!Resolution}
The resolution of STS is determined by the range of energies electrons have when contributing to the tunneling process. This range is $eV$. When $T>0$ the DOS is smeared out and is described by $$f(E)=\frac{1}{1+exp\left(\frac{E-E_F}{kT}\right)}$$ 
Since electrons from occupied states (DOS is Fermi distributed) tunnel into unoccupied states the transmission function has the structure $$T(E,eV,T)=T(E,eV)f(E)[1-f(eV-E)]$$ 
When looking at the shape of the Fermi distribution one can see that most of the electrons participating in the tunneling process arise from a rather narrow area around the Fermi level of the negatively biased electrode (broadening of fermi edge at $T=300K\hat=\SI{0.026}{\eV}$. Electron distribution of tip and sample are broadened by $2 k_b T=\SI{0.054}{\eV}$ thus the energetic range where electrons may come from is \SI{0.1}{\eV}. From the uncertainty relation $\Delta x \Delta k \geq 1/2$ and the dispersion relation for metals $$ \Delta E\ge \frac{\hbar^2k_F}{2M^*\Delta x}=0.47\frac{E_F-E_0}{rk_F} $$\cite{chen_introduction_2008}. ``The asymmetric form of $T(E,eV)$, with the sharp inscrease at $E_F$, helps to make the effective resolution of the STM somewhat higher when probing empty states of the sample than when probing filled states.''
The resolution at room temperature is \SI{140}{\m\eV}\cite{hansma_tunneling_1982}.
As the tunneling transmission is always a factor of the tip and sample DOS, STS is always limited to the unknown electronic structure of the tip. While geometry at the tip apex is successfully enhanced with field evaporation techniques its electronic structure may differ greatly from the bulk one due to unusual bonding geometry and small size. Some\cite{tersoff_role_1990,ciraci_tip-sample_1990,lawunmi_theoretical_1990,kobayashi_simulation_1990} groups have calculated band structures for different tips and their influence in the tunneling process.


\paragraph{Other methods with STM}Inelastic tunneling => vibrational states resolvable => chemical descicive measurements; Tip induced band bending

\paragraph{Gundlach oscillations}
Gundlach was the first who calculated transmission currents for trapezial potential barriers \cite{gundlach_zur_1966}. The oscillations named after him are due to standing wave states in the potential tip-sample potential barrier \cite{binnig_tunneling_1985,becker_electron_1985}
``When the Fermi level of the tip is close to the vacuum level of  the  sample,  the  contribution  of  the  image  potential  is significant. The superposition of the  image  potential  and the electrostatic  potential forms a specific potential well, and the lowest-order peak is a Gundlach oscillation related to a standing-wave state in this well. When the Fermi level of the tip is higher than the vacuum level of the sample, the image potential becomes negligible, and the potential well can be  approximated  by a triangular  shape. Those peaks beyond the lowest-order peak are the Gundlach oscillations related to the standing-wave states in the triangular well. Derivation  based  on  quantum  mechanics  shows  that  the energy difference of the standing-wave states in the triangular  well  is  proportional  to $F^{2/3}$,  where F is  the electric field in the tip-sample gap''\cite{lin_manifestation_2007}